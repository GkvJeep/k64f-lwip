\capitulo{3}{Conceptos teóricos}
En este capítulo se sintetizan algunos de los aspectos tratados en este proyecto
para mejorar su compresión y entendimiento.

\section{Sistemas empotrados}
Parte importante del proyecto se centra en obtener un sistema empotrado capaz de
comunicarse usando los protocolos TCP/IP. A continuación se describen los
conceptos más relevantes en torno a los SE.

\subsection{Descripción}
Se puede considerar que un sistema empotrado es aquel cuyo
\extranjerismo{hardware} y \extranjerismo{software} se encuentran estrechamente
relacionados, está diseñado para cumplir con una función específica, se haya
integrado en un sistema mayor, no se espera que el usuario lo modifique y puede
trabajar sin interacción o con la mínima interacción humana necesaria.
\cite{jime13}

\imagen{refrigerator}{Ejemplo de integración de SE. \cite{nxp01}}

En la figura \ref{fig:refrigerator} se muestra un ejemplo de uso de varios SE
dentro de un sistema mayor, en este caso un frigorífico inteligente. El sistema
cuenta con varios componentes, la interfaz de usuario, la gestión del sistema,
el control del motor o la conectividad. Cada uno de los componentes se ayuda de
un SE para realizar su función asignada.

Los SE cuentan con el \extranjerismo{hardware} específico para la tarea a
realizar. La interfaz de usuario puede contar con pantallas o botones. La
gestión del sistema tiene acceso a sensores, el control de la corriente, la
iluminación o los ventiladores. El control del motor presenta componentes
eléctricos para la regulación del compresor. Un módulo de energía, un triodo
para corriente alterna o un módulo de corrección del factor de potencia son
componentes que pueden estar presentes en el control del motor. Asimismo, el
\extranjerismo{software} ejecutado en cada uno de estos SE varía según la
función a desempeñar realizando únicamente las tareas necesarias.

Por otra parte, también se puede advertir que los SE no están pensados para que
el usuario los modifique o programe, ni requieren que la interacción sea
constante por parte del usuario para su correcto funcionamiento.

Los SE y los sistemas que los emplean se encuentran fácilmente. Se hallan en 
sistemas de movilidad y transporte, automatización industrial, sector sanitario,
edificios inteligentes, redes de suministro inteligentes, investigación
científica, seguridad pública, supervisión de salud estructural, recuperación de
desastres, robótica, agricultura y ganadería, aplicaciones militares,
telecomunicaciones y electrónica de consumo. \cite{marw18}

\subsection{Características del \extranjerismo{hardware}}
Los SE disponen de componentes \extranjerismo{hardware} que siendo específicos
para la tarea a la que están destinados se pueden generalizar en procesador,
memoria y puertos de entrada y salida. El procesador se encarga de ejecutar las
instrucciones de los programas que manejan las entradas y las salidas del
sistema. Los programas ejecutados y los datos generados se almacenan en la
memoria. Y los puertos de entrada y salida, envían y reciben la señales con las
que trabaja el procesador.\cite{jime13}

También existen otros elementos que se encuentran a menudo en los SE:
\begin{itemize}
    \item Puertos de comunicación serie o paralelo
    \item Dispositivos de interfaz humana
    \item Sensores
    \item Actuadores
    \item Conversores analógica-digital (ADC)
    \item Conversores digital-analógica (DAC)
    \item Componentes de diagnóstico y redundancia
    \item Componentes de apoyo al sistema
    \item Otros subsistemas:
    \begin{itemize}
        \item Circuito integrado para aplicaciones específicas (ASIC)
        \item Matriz de puertas programables (FPGA)
    \end{itemize}
\end{itemize}

En la figura \ref{fig:diagr_bloques_hw} se presenta un diagrama de bloques
en el que se puede observar de forma general los componentes que forman un SE
y su funcionamiento. Las entradas son procesadas por el microcontrolador que a
su vez generará las salidas apropiadas. 
\imagen{diagr_bloques_hw}{Diagrama de bloques del HW de un SE}

\subsection{Características del \extranjerismo{software}}
En cuanto al \extranjerismo{software} presente en los sistemas embebidos, se
pueden diferenciar varios grupos. El primero de ellos los
\extranjerismo{drivers} o controladores encargados de la interacción directa
con el \extranjerismo{hardware} del SE.

Luego se encuentra el \extranjerismo{middleware}, que es aquel
\extranjerismo{software} que ocupa una posición entre el sistema operativo y los
programas. El término se usa con frecuencia en librerías de rutinas de
infraestructura que proporcionan servicios a los desarrolladores de los
programas. \cite{butt16}

Junto a lo anterior se puede hallar un Sistema operativo en tiempo real (RTOS).
En un sistema operativo de propósito general varias tareas se ejecutan de forma
aparentemente simultánea. De este modo se puede repartir el tiempo de ejecución
de manera equitativa entre usuarios, por ejemplo. En cambio, en un RTOS se prima
la ejecución de las tareas en un tiempo estrictamente limitado. Con un sistema
de prioridades se determina la importancia de las tareas y cuales necesitan ser
realizadas sin demora. \cite{amaz19}

Por último y funcionado sobre lo anterior se encuentra los programas necesarios
para el funcionamiento del SE. En caso de contar con un RTOS será este el que se
encargue de ejecutar una tarea u otra. Sino, de forma conocida como
\extranjerismo{bare-metal} se ejecutan las tareas de forma secuencial. Secuencia
que solo es alterada en caso surgir una interrupción.

\imagen{diagr_bloques_sw}{Digrama de bloques del SW de un SE}
En el diagrama de bloques de la figura \ref{fig:diagr_bloques_sw} se observan
como se ubican unos componentes sobre otros. Los componentes se comunican con
aquellos adyacentes. El RTOS es un componente opcional y que funciona de forma
paralela a \extranjerismo{drivers} y \extranjerismo{middleware}.
