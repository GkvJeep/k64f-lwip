\apendice{Especificación de Requisitos} \label{ch:especificacion}

\section{Introducción} \label{sec:spec-intro}
En este apéndice de especificación de requisitos del \extranjerismo{software}
se proporciona una descripción completa del propósito y funcionalidad del
\extranjerismo{software}. Se detallan las funciones que debe realizar el
\extranjerismo{software} y se muestran casos de uso de como los usuarios
utilizarán el \extranjerismo{software}.

También sirve para definir como tiene que interactuar el 
\extranjerismo{software} con el \extranjerismo{hardware} y con otros
\extranjerismo{softwares}. Además, sirve para definir otros requisitos de
carácter no funcional.

Para realizar la especificación de requisitos del \extranjerismo{software}
The Institute of Electrical and Electronics Engineers (IEEE) ha publicado
varios estándares al respecto, IEEE 830-1998 \cite{webpage:ieee830-1998}
y IEEE 29148-2011 \cite{webpage:ieee29148-2011} \footnote{Actualizado
recientemente a IEEE 29148-2018 \cite{webpage:ieee29148-2018}.}.
La especificación de requisitos del \extranjerismo{software} se realizará
siguiendo algunas de las indicaciones de estos estándares.

\subsection{Identificación} \label{sec:spec-id}
Este apéndice sirve para especificar los requisitos del \extranjerismo{software}
realizado en el proyecto. Es decir:
\begin{itemize}
  \item El \extranjerismo{software} utilizado por un sistema empotrado para 
  realizar transmisiones TCP/IP.
  \item El \extranjerismo{software} utilizado por la aplicación web para 
  realizar transmisiones TCP/IP con destino al sistema.
\end{itemize}

Está previsto que ambos \extranjerismo{softwares} se publiquen con la versión
1.0 y en este documento se describe su funcionalidad.

\subsection{Audiencia destinataria}
\label{sec:spec-audiencia}
Como audiencia destinataria de este documento se incluye cualquier persona
interesada en el proyecto. Tanto tutor, evaluadores, usuarios, como futuros
desarrolladores interesados en conocer la funcionalidad del
\extranjerismo{software} del sistema empotrado y del \extranjerismo{software}
de la aplicación web.

\subsection{Abreviaciones} \label{sec:spec-abreviaciones}
\begin{description}
  \item[Sistema empotrado:] SE
  \item[\extranjerismo{Software} del sistema empotrado:] SW del SE
  \item[Aplicación web:] AW
  \item[\extranjerismo{Software} de la aplicación web:] SW de la AW
  \item[Requisito de interfaces externas:] RE 
  \item[Requisito funcional:] RF
  \item[Requisito no funcional:] RNF
\end{description}

\section{Objetivos generales}
El proyecto cuenta con los siguientes objetivos en relación al funcionamiento
del SE y de la AW.

\begin{itemize}
  \item Configurar un SE que sea capaz de conectarse en red usando TCP/IP.
  \item El SW del SE puede realizar varias acciones sobre el
        \extranjerismo{hardware}.
  \item El SW de la AW puede comunicarse con el SE enviando comandos via TCP/IP.
  \item El SW de la AW puede ordenar la realización de una acción sobre el
        \extranjerismo{hardware}.
\end{itemize}

\subsection{Clases de usuario y características} \label{sec:spec-usuarios}
Cualquier persona interesada en SE puede ser usuaria del SE y de la AW.
Además, no se requieren conocimientos avanzados ni experiencia técnica
para su uso. Bastará con conectar el SE y acceder a la AW.

\subsection{Entorno operativo} \label{sec:spec-entorno}
Por un lado, el SE require de un servidor DHCP para obtener una dirección IP
que le permita comunicarse con el resto de los dispositivos de la red.

Por el otro, para acceder a la AW solo se requiere de un navegador sin importar
si es un equipo de sobremesa o en un dispositivo móvil. Sin embargo,
si el usuario desea ejecutar la AW por si mismo, se requiere de un sistema
que cuente con un servidor de aplicaciones como GlassFish.

\subsection{Documentación del usuario} \label{sec:spec-docs}
Junto con el \extranjerismo{software} se proporciona la memoria que describe
el desarrollo y sus anexos con información complementaria. A destacar, el
apéndice con el manual de usuario que cuenta con indicaciones para el uso del
\extranjerismo{software}.

\section{Catalogo de requisitos}

\subsection{Requisitos de interfaces externas} \label{sec:spec-interfaces}
\subsubsection{Interfaces de usuario} \label{sec:spec-interfaces-usuario}
\begin{itemize}
  \item \textbf{RE-1 Acceso web} El usuario debe poder interactuar con el SE
  mediante una interfaz web.
\end{itemize}

\subsubsection{Interfaces de \extranjerismo{hardware}}
\label{sec:spec-interfaces-hw}
\begin{itemize}
  \item \textbf{RE-2 Acceso a la red} El SE tiene que ser capaz de usar
  Ethernet. RE de alta prioridad.
  \item \textbf{RE-3 Transmisiones en red} El SE tiene que ser capaz de usar
  TCP/IP. RE de alta prioridad.
  \item \textbf{RE-4 Uso de LED} El SE tiene que ser capaz de usar LED. RE de
  alta prioridad.
  \item \textbf{RE-5 Uso de I\textsuperscript{2}C} El SE tiene que ser
  capaz de usar I\textsuperscript{2}C. RE de alta prioridad.
  \item \textbf{RE-3 Uso de PWM} El SE tiene que ser capaz de usar PWM. RE de
  alta prioridad.
\end{itemize}

\subsubsection{Requisitos funcionales del SW del SE}
\label{sec:spec-funcionales-se}
\begin{itemize}
  \item \textbf{RF-1 Recepción de comandos} El SE tiene que recibir comandos
  transmitidos mediante paquetes TCP con destino a su correspondiente dirección
  IP y puerto TCP abierto. RF de alta prioridad.
  \item \textbf{RF-2 Identificación de comandos} El SE tiene que se capaz de
  identificar los comandos recibidos para poder ejecutar las acciones correctas.
  RF de alta prioridad.
  \item \textbf{RF-3 Acción sobre los LED RGB} El SE tiene que poder variar
  los colores producidos por los LED RGB. RF de alta prioridad.
  \item \textbf{RF-4 Acción sobre el LCD} El SE tiene que poder mostrar
  cadenas de caracteres en el LCD. RF de alta prioridad.
  \item \textbf{RF-5 Acción sobre los LED PWM} El SE tiene que poder regular
  la intensidad del brillo de los LED mediante PWM. RF de alta prioridad.
\end{itemize}

\subsubsection{Requisitos funcionales del SW de la AW}
\label{sec:spec-funcionales-aw}
\begin{itemize}
  \item \textbf{RF-6 Selección del SE} El usuario debe poder seleccionar el SE 
  con el que desea interactuar. La selección se realiza introduciendo la
  dirección IP y el puerto del SE escogido. RF de alta prioridad.
  \item \textbf{RF-7 Envío de comandos} La aplicación tiene que ser capaz de
  construir comandos inteligibles por el SE y que realicen las operaciones
  indicadas por el usuario. RF de alta prioridad.
  \item \textbf{RF-8 Selección del color} El usuario debe poder escoger un color
  a ser exhibido por los LED RGB. RF de alta prioridad.
  \item \textbf{RF-9 Introducción de mensajes} El usuario debe poder introducir
  las cadenas de caracteres a mostrar por el LCD. RF de alta prioridad.
  \item \textbf{RF-10 Regulación del brillo} El usuario debe poder regular la
  intensidad del brillo de los LED PWM. RF de alta prioridad.
\end{itemize}

\subsubsection{Requisitos no funcionales del SE y de la AW}
\label{sec:spec-nofuncionales}
\begin{itemize}
  \item \textbf{RNF-1 Rendimiento del SE} El SE tiene que ser capaz de realizar
  las acciones ordenas por el usuario con prontitud. RNF de media prioridad.
  \item \textbf{RNF-2 Seguridad del SE} El SE tiene que asegurar que sus
  componentes no presentan riesgos eléctricos al usuario. RNF de alta prioridad.
  \item \textbf{RNF-3 Calidad del SW} El SW tiene que garantizar cierto nivel
  de calidad, p. ej., incluyendo comentarios que faciliten su comprensión para
  un mantenimiento o portabilidad posteriores. RNF de media prioridad.
\end{itemize}

\section{Especificación de requisitos}
