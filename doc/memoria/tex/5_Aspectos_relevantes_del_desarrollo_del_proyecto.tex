\capitulo{5}{Aspectos relevantes del desarrollo del proyecto}
{\label{ch:aspectos}}
Durante el desarrollo del proyecto han surgido diversos hechos relevantes. En
este capítulo se comentan los eventos ocurridos y la decisiones tomadas al
respecto.

\section{Aprendizaje}{\label{sec:aprendizaje}}
Una faceta fundamental del proyecto ha sido el aprendizaje. Conocer nuevas 
herramientas, técnicas y metodologías, profundizar en las ya conocidas y revisar
aquellas que hubieran podido caer en el olvido.

Revisando los objetivos personales \ref{sec:obj_personales} se puede ver como
el deseo de aprendizaje está presente al dejar patente que no solo se quiere
cumplir con los objetivos generales \ref{sec:obj_generales} y técnicos 
\ref{sec:obj_tecnicos}.

Querer utilizar nuevas herramientas, técnicas, metodologías, etc. no está exento
de problemas o contratiempos que no dudaron en dejarse ver desde el inicio.

\section{Inicio}{\label{sec:inicio}}
Para desarrollar el \extranjerismo{software} que va a utilizar la placa K64F
es necesario usar un IDE. En la sección sobre los IDE valorados \ref{sec:ide}
se comentan brevemente las diferencias entre Kinetis Design Studio (KDS) y
MCUXpresso.

La decisión de usar MCUXpresso estuvo basada fundamentalmente en el abandono
por parte de NXP de KDS. Esto no quiere decir que no se pudiera haber usado KDS.
Pero siguiendo la propia recomendación del fabricante y queriendo usar
herramientas modernas se tomó la decisión de usar MCUXpresso.

La principal diferencia entre IDE es la presencia de Processor Expert (PE) en
KDS y la respectiva ausencia en MCUXpresso. Esta herramienta fue la usada en su
momento para aprender a desarrollar SE. Al perder PE fue necesario dedicar
tiempo a aprender como usar las nuevas herramientas presentes en la
\extranjerismo{suite} de MCUXpresso, el IDE, el SDK y las Config Tools.

La primera toma de contacto estuvo apoyada en los ejemplos ofrecidos por el SDK.
Son pequeñas piezas de código que realizan funciones simples que permiten
experimentar con la placa y añadir modificaciones para ver como responde.

Familiarizado con el IDE, tocaba aprender a configurar los relojes del sistema.
Usar como base la configuración de los ejemplos sirvió para conocer que relojes
se tenían que activar, a que frecuencia y que salidas se tenían que habilitar.

Por último queda configurar el MCU y sus pines. Para poder usar un componente
hay que decir al MCU que pines se van a utilizar y la función deseada. Que un
dispositivo funcione correctamente requiere de la sección de los pines 
correctos y de la configuración eléctrica correcta. Por ejemplo,
para poder usar el puerto Ethernet no sirve la configuración por defecto sino
que hay que configurar el estado \extranjerismo{pull-up} de alguno de sus pines.

Dominado MCUXpresso se pudo comenzar el desarrollo propiamente dicho.

\section{Desarrollo con la placa K64F}{\label{sec:desarrollo-k64f}}

\subsection{Técnicas y metodologías}{\label{sec:desarrollo-tym}}
En esta fase se tomó la decisión junto con el tutor, de integrar metodologías y 
técnicas que podían resulta beneficiosas para el proyecto.

Una medida tomada fue usar GitHub como repositorio del código y sistema de 
control de versiones. Además, como los IDE y VS Code están preparados para
usar Git, su uso resultó bastante asequible y cómodo.

Otra medida tomada fue usar Scrum y para ello el trabajo se planificó en
sucesivos \extranjerismo{sprints}. Scrum propone equipos de desarrollo
de varias personas, reuniones diarias, reuniones al terminar el
\extranjerismo{sprint}, el rol de Scrum Master, el Product Owner, etc.
Como el proyecto se ha realizado individualmente, usar Scrum ha servido como
forma de conocer la metodología, usando principalmente el \extranjerismo{sprint}
como herramienta para dividir el trabajo y planificar el proyecto.

\subsection{Conexión a la red}{\label{sec:desarrollo-red}}
Con la placa siendo capaz de usar el puerto Ethernet se presentaba el asunto
de la obtención del dirección IP. Establecer una dirección fija presenta
problemas si ya está usada o si se conecta en una subred con diferente rango
de direcciones.

Por este motivo se tomó la decisión de usar DHCP. Esta decisión
obliga a contar con un servidor DHCP en la misma red a la que se conecta la
placa. De no obtener dirección la placa se quedaría a la espera indefinidamente.
Para que el usuario sepa la dirección de la placa, en un primer momento se
mostraba en la consola de depuración para más tarde mostrar también en el LCD
tanto la dirección IP como el puerto TCP a la escucha.

Con la pila TCP/IP en funcionamiento era posible enviar paquetes de datos a la
placa. Como lwIP implementa varios protocolos, ICMP entre ellos, era posible
realizar \extranjerismo{pings} para comprobar el estado de la placa.

\subsection{Funciones del \extranjerismo{hardware}}{\label{sec:desarrollo-hw}}
Para ejemplificar el uso de un SE conectado en red se decidió aprovechar el
\extranjerismo{hardware} incluido en la placa. Representando a unos actuadores y 
siendo los componentes más visibles de la placa se decidió utilizar los LED de
colores RGB. Para poder activar o desactivar los LED se determinó enviar un
comando con una instrucción y argumento específicos para el LED a alterar. El
comando se transmitiría dentro un paquete TCP.

El parámetro de cada comando sirve para indicar el color a alterar.
Los comandos para encender los tres colores básicos de los LED RGB son:
\begin{quotation}
  ``led:r''

  ``led:g''

  ``led:b''
\end{quotation}

Como los LED se pueden encender simultáneamente se puede combinar su luz para
crear nuevos colores. En la tabla siguiente se pueden ver los todos 
colores posibles. 

\tablaSmall{Colores producibles usando los LED RGB}
{l c c c}{coloresrgb}
{\multicolumn{1}{l}{Color} & LED Rojo & LED Verde & LED Azul\\}
{
  Rojo     & X &   &   \\
  Verde    &   & X &   \\
  Azul     &   &   & X \\
  Amarillo & X & X &   \\
  Magenta  & X &   & X \\
  Cyan     &   & X & X \\
  Blanco   & X & X & X \\
  Ninguno  &   &   &   \\
}

El LCD se incluyó con la intención de mejorar la comunicación del usuario y
ampliar la funcionalidad del SE. Esta decisión provocó inconveniente que volvía
a involucrar el uso de MCUXPresso. Con anterioridad para poder utilizar el LCD
se usaba una librería de funciones adaptada para usar los componentes generados
por PE. 

En origen la librería ya había sido adaptada desde código para Arduino a
código para PE. Así que surgió la necesidad de adaptar de nuevo la librería para
usar el código generado por MCUXpresso. Como la librería usaba funciones de
espera también se tuvieron que portar estas funciones.

El comando a transmitir indica que la operación se trata de un mensaje para la
pantalla, la línea en la que mostrar el mensaje y la cadena de caracteres a
presentar.
Ejemplos de comandos para mostrar una cadena de caracteres en cada línea:
\begin{quotation}
  ``msg:0:Lorem ipsum''

  ``msg:1:dolor sit amet''
\end{quotation}

La última función añadida fue modificar la intensidad del brillo de los LED
incorporados en la placa de expansión. La regulación de la intensidad se
consigue usando los pines compatibles con PWM del MCU. Los pines PWM no
están comunicados con los LED RGB de la propia placa K64F así que se conectaron
a los pines de los LED de placa de expansión.

Al transmitir un comando y tras indicar que es una instrucción de tipo PWM,
los argumentos indican el LED a regular y la intensidad (de 0\% a 100\%)
deseada. Por ejemplo, los siguientes comandos sirven para regular la intensidad
de cada LED con incrementos del 25\%.
\begin{quotation}
  ``pwm:w:0''

  ``pwm:g:25''

  ``pwm:y:50''

  ``pwm:r:100''
\end{quotation}

\subsection{Montaje \extranjerismo{hardware}}{\label{sec:desarrollo-montaje}}
Una vez que cada componente funcionaba de la manera esperada se realizó
el montaje en las placas de pruebas para poder hacer funcionar el SE al
completo. 

\subsection{Tareas del RTOS}{\label{sec:desarrollo-rtos}}
Como se ha visto en la sección sobre \extranjerismo{software} de los SE 
\ref{sec:se-sw}, es habitual contar con un RTOS encargado de la planificación
de las tareas ejecutadas. El RTOS a usar fue FreeRTOS \ref{sec:otros} por su
conocimiento y experiencia previa. Para cada función disponible en la placa se
creó una tarea concreta. Es decir, hay una tarea para usar el LCD, otra para los
LED de la placa de expansión y otra para los LED RGB de la propia placa. 
Las tareas quedan a la espera de recibir un comando de la tarea, de mayor
prioridad, que está escuchando en la dirección y el puerto asignados.

\subsection{Desarrollo de la aplicación web}{\label{sec:desarrollo-app}}
Con la placa operativa llegaba el turno de crear una aplicación web que
permitiese actuar remotamente con la placa. La tecnología para crear la
aplicación web es JSF, la parte encargada de comunicarse con la placa escrita
en Java y la página visible usada por el usuario en XHTML y CSS.

Al desconocer CSS se presentaba una oportunidad de aprender sobre ese lenguaje.
De nuevo requería dedicar cierto tiempo al estudio y aprendizaje hasta lograr
una página con un aspecto aceptable.

Al usuario se le muestran cuatro apartados. El primero le permite introducir la
dirección IP y el puerto TCP mostrados por el LCD. En caso de existir varios SE
se pueden cambiar estos parámetros para escoger el SE con el que interactuar.

El siguiente bloque muestra un cuadrado por cada uno de los colores que se
pueden generar con los LED RGB. Si lo que se desea es apagar los LED, pulsando
el cuadrado negro se apagan todos.

El tercer apartado permite introducir una cadena de texto que será enviada al
LCD. Como la pantalla solo puede mostrar 16 caracteres los cuadros de texto
están limitados a esta longitud. Se muestran dos cuadros, uno para cada línea
del LCD.

El último bloque está compuesto por cuatro controles deslizantes. Hay uno por
LED de la placa de expansión. Usando estos controles se puede ajustar de forma
visual la intensidad, yendo de 0\% a 100\% deslizando el control de izquierda
a derecha.

Como todo el contenido se muestra en una página era necesario modificar el
comportamiento realizado tras pulsar los botones. Por defecto, al pulsar un
botón la página se desplazaba hasta arriba. El inconveniente se solucionó
inyectando código HTML en los botones para modificar su comportamiento.

\section{Documentación}{\label{sec:documentacion}}
En cuanto a la documentación, más allá de aprender a utilizar \LaTeX{} no se
presentaron mayores inconvenientes. Aparecieron las dudas habituales que surgen
cuando se está aprendiendo a usar nuevas herramientas o técnicas pero se fueron
solventando a medida que fueron surgiendo.

El uso de la plantilla facilitada por el tutor allanó la creación y edición de 
los documentos de la memoria y anexos. Se realizaron pequeñas modificaciones a
la plantilla con la intención de mejorar el resultado obtenido. 

Uno de los cambios fue usar el paquete \emph{biblatex-iso690} para generar la
bibliografía de acuerdo a la norma UNE-ISO 690:2013. Otra adición fue el
paquete \emph{url} que se encarga de mejorar la forma en que se tratan las URL
en el texto.
