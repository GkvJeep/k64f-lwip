\apendice{Documentación técnica de programación} \label{ch:man-dev}

\section{Introducción} \label{sec:man-dev-intro}
Para que cualquier persona interesada en conocer, modificar o extender
el sistema desarrollado, este apéndice muestra los aspectos a tener en cuenta
para lograrlo.

Como los componentes del sistema, el sistema empotrado y la aplicación web,
tienen características tan diferentes cada uno necesita herramientas específicas
para su desarrollo.

Se detallan las particularidades de cada herramienta, los ajustes de
configuración, la estructura que toman los directorios con los ficheros
y los pasos necesarios para poner en funcionamiento el proyecto.



\section{Estructura de directorios} \label{sec:man-dev-struct}
Para empezar a trabajar con el proyecto lo primero que se necesita es una
copia del código fuente. La manera más sencilla de consultar y obtener el código
es acudir a los repositorios de GitHub. Hay un repositorio dedicado al \sw{}
del sistema empotrado \cite{webpage:repo-se} y otro para el \sw{} de la
aplicación web \cite{webpage:repo-aw}.


\subsection{Estructura de directorios del sistema empotrado}
\label{sec:man-dev-struct-se}
El código fuente del \sw{} del sistema empotrado toma la siguiente estructura:

\begin{description}
  \item[/] Directorio raíz, contiene el resto de directorios. Incluye el
  fichero LICENSE que contiene los términos y condiciones del licenciamiento
  del \sw{}. Además contiene un fichero de tipo ``mex'' generado por el entorno
  de desarrollo (IDE) que sirve para configurar el \hw{} de la placa de
  desarrollo. Y el fichero Doxyfile que configura la herramienta de documentación.
  \item[/CMSIS/] Fuentes pertenecientes al Cortex Microcontroller Software
  Interface Standard (CMSIS). Proporciona interfaces al procesador y sus
  periféricos.
  \item[/amazon-freertos/] Fuentes pertenecientes a FreeRTOS, el RTOS usado por
  el sistema empotrado.
  \item[/board/] Fuentes autogenerados por el IDE y sus Config Tools que 
  permiten habilitar y configurar el \hw{} de la placa de desarrollo.
  \item[/doc/] Contiene la documentación del código y del proyecto.
  \item[/doc/amazon-freertos/] Documentación incluida con FreeRTOS.
  \item[/doc/lwip/] Documentación incluida con lwIP.
  \item[/doc/memoria/] Documentación del proyecto, la memoria descriptiva junto
  con sus apéndices.
  \item[/doc/html] Documentación del código fuente generada por Doxygen.
  \item[/drivers/] Fuentes con los controladores necesarios para trabajar con el
  \hw{}.
  \item[/lwip/] Fuentes relativos a lwIP, la implementación de la pila de 
  protocolos TCP/IP.
  \item[/source/] Código fuente del proyecto. A destacar, el fichero ``main.c''
  encargado del funcionamiento general del sistema. Están presentes los ficheros
  encargados de tratar con \hw{}. También se encuentran los ficheros de
  configuración.
  \item[/startup/] Código de arranque generado por el IDE.
  \item[/utilities/] Código generado por el IDE con utilidades auxiliares usadas
  para depuración o registro de eventos.
\end{description}


\subsection{Estructura de directorios de la aplicación web}
\label{sec:man-dev-struct-aw}
El código fuente del \sw{} de la aplicación web toma la siguiente estructura:

\begin{description}
  \item[/] Directorio raíz, contiene el resto de directorios. Aquí se ubica el
  fichero ``pom.xml'' usando por la herramienta Maven para la gestión y
  construcción del proyecto.
  \item[/doc/] Documentación del código fuente generada por Javadoc.
  \item[/src/main/] Contiene todo el código perteneciente a la aplicación web.
  \item[/src/main/java/] Código Java que implementa la lógica de negocio.
  \item[/src/main/webapp/] Contiene el fichero XHTML con el código de la
  interfaz web.
  \item[/src/main/webapp/WEB-INF] Contiene los ficheros XML usados por
  el servidor de aplicaciones.
  \item[/src/main/webapp/resources/] Recursos adicionales de la interfaz web.
  \item[/src/main/webapp/resources/css/] Contiene el fichero CSS que modifica el
  aspecto de la interfaz.
  \item[/src/main/webapp/resources/images/] Contiene las imágenes y otros
  recursos gráficos a usar en la interfaz.
\end{description}



\section{Manual del programador}
Para desarrollar el proyecto se han utilizado dos IDE diferentes, uno para cada
\sw{}. Aunque no es indispensable usar las mismas herramientas, a continuación
se muestra como obtener, instalar, configurar y utilizarlas de la misma
manera que se ha hecho durante el proyecto.


\subsection{MCUXpresso IDE} \label{sec:man-dev-mcuxpresso}
El IDE se puede obtener gratuitamente desde la zona de Recursos para
desarrolladores de NXP \cite{webpage:mcuxpresso}. El único requisito establecido
por NXP es tener una cuenta registrada, gratuita también, en su sitio web.

Una vez iniciada la sesión y accedido al área de descarga se puede obtener
la versión más reciente del IDE. De ser necesario, también es posible descargar
una versión antigua desde la pestaña \extranjerismo{Previous}.

Tras aceptar los términos y condiciones de uso se muestran los enlaces de
descarga de los instaladores del IDE. 

\imagen{mcu_descarga}{Descarga de MCUXpresso IDE}

El instalador sigue el proceso de instalación habitual a la mayoría de programas.
Aceptar licencia de uso, indicar la ubicación del programa e informar
que se van a instalar controladores de depuración son los pasos más relevantes.

\cleardoublepage
\section{Compilación, instalación y ejecución del proyecto}

\section{Pruebas del sistema}
