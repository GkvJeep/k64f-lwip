\capitulo{3}{Conceptos teóricos}{\label{ch:conceptos}}
En este capítulo se sintetizan los aspectos esenciales tratados en este proyecto
para mejorar su compresión y entendimiento.



\section{Sistemas empotrados}{\label{sec:se}}
Parte importante del proyecto se centra en obtener un sistema empotrado (SE)
capaz de comunicarse usando los protocolos TCP/IP. A continuación se describen
los conceptos más relevantes en torno a los SE.


\subsection{Descripción}{\label{sec:se-desc}}
Se puede considerar que un sistema empotrado es aquel cuyo
\extranjerismo{hardware} y \extranjerismo{software} se encuentran estrechamente
relacionados, está diseñado para cumplir con una función específica, se
encuentra integrado en un sistema mayor, no se espera que el usuario lo
modifique y puede trabajar sin interacción o con la mínima interacción humana
necesaria \cite{jimenez2014}.

\imagen{refrigerator}{Ejemplo de integración de SE.
\cite{webpage:nxp-refrigerator}}

En la figura \ref{fig:refrigerator} se muestra un ejemplo de aplicación de
varios SE dentro de un sistema mayor, en este caso un frigorífico inteligente.
El sistema cuenta con varios componentes: la interfaz de usuario, la gestión del
sistema, el control del motor o el módulo de conectividad. Cada uno de los
componentes se ayuda de un SE para realizar su función asignada.

Cada SE utiliza el \extranjerismo{hardware} específico para realizar su función
correspondiente.
\begin{itemize}
  \item La interfaz de usuario cuenta con pantallas o botones.
  \item La gestión del sistema tiene acceso a sensores, el control de la
  corriente, la iluminación o los ventiladores.
  \item El control del motor presenta componentes eléctricos encargados de la
  regulación del compresor: un módulo de energía, un triodo para corriente
  alterna o un módulo de corrección del factor de potencia.
\end{itemize}

Del mismo modo, el \extranjerismo{software} ejecutado por cada SE estará
diseñado e implementado según la función a desempeñar realizando únicamente
tareas específicas a su función.

Por otra parte, también se puede advertir que los SE no están pensados para que
el usuario los modifique o programe, ni requieren que la interacción sea
constante por parte del usuario para su correcto funcionamiento.

Los SE y los sistemas que los emplean se encuentran en todas partes. Se hallan
en sistemas de movilidad y transporte, automatización industrial, sector
sanitario, edificios inteligentes, redes de suministro inteligentes,
investigación científica, seguridad pública, supervisión de salud estructural,
recuperación de desastres, robótica, agricultura y ganadería, aplicaciones
militares, telecomunicaciones y electrónica de consumo \cite{marwedel2018}.


\subsection{Características del \extranjerismo{hardware}}{\label{sec:se-hw}}
Aunque el \extranjerismo{hardware} de los SE se adapta a la función que
desempeñan, todos ellos cuenta con un microcontrolador (MCU) encargado de
controlar las operaciones del SE. Un MCU está compuesto por un procesador,
memoria --- RAM y ROM --- y puertos de entrada y salida. El procesador se
encarga de ejecutar las instrucciones de los programas que manejan las entradas
y las salidas del sistema. Los programas ejecutados y los datos generados se
almacenan en la memoria. Y los puertos de entrada y salida, envían y reciben la
señales con las que trabaja el procesador\cite{jimenez2014}.

También existen otros elementos que se encuentran a menudo en los SE:
\begin{itemize}
    \item Puertos de comunicación serie o paralelo
    \item Dispositivos de interfaz humana
    \item Sensores
    \item Actuadores
    \item Conversores analógica-digital (ADC)
    \item Conversores digital-analógica (DAC)
    \item Componentes de diagnóstico y redundancia
    \item Componentes de apoyo al sistema
\end{itemize}

En la figura \ref{fig:diagr_bloques_hw} se presenta un diagrama de bloques
en el que se pueden observar los componentes que forman un SE de forma general
y su dirección en el funcionamiento. Las entradas son procesadas por el
microcontrolador que a su vez generará las salidas apropiadas.
\imagen{diagr_bloques_hw}{Diagrama de bloques del HW de un SE}


\subsection{Características del \extranjerismo{software}}{\label{sec:se-sw}}
En cuanto al \extranjerismo{software} presente en los sistemas embebidos, se
pueden diferenciar varios grupos. El primero de ellos los
\extranjerismo{drivers} o controladores encargados de la interacción directa
con el \extranjerismo{hardware} del SE.

Luego se encuentra el \extranjerismo{middleware}, que es aquel
\extranjerismo{software} que ocupa una posición entre el sistema operativo y los
programas. El término se usa con frecuencia en librerías de rutinas de
infraestructura que proporcionan servicios a los desarrolladores de los
programas. \cite{butterfield2016}

Junto a lo anterior se puede hallar un Sistema operativo en tiempo real (RTOS).
En un sistema operativo de propósito general varias tareas se ejecutan de forma
aparentemente simultánea. De este modo se puede repartir el tiempo de ejecución
de manera equitativa entre usuarios, por ejemplo. En cambio, en un RTOS se prima
la ejecución de las tareas en un tiempo estrictamente limitado. Con un sistema
de prioridades se determina la importancia de las tareas y cuales necesitan ser
realizadas sin demora. \cite{webpage:amazon-rtos}

Por último y funcionado sobre lo anterior se encuentra los programas necesarios
para el funcionamiento del SE. En caso de contar con un RTOS será este el que se
encargue de ejecutar una tarea u otra. Sino, existen otras arquitecturas que
controlan el orden y la prioridad de ejecución: bucle infinito, bucle infinito 
interrupciones, de bajo consumo o con planificador. Su elección dependerá de los
requisitos impuestos al sistema empotrado.

\imagen{diagr_bloques_sw}{Digrama de bloques del SW de un SE}

En el diagrama de bloques de la figura \ref{fig:diagr_bloques_sw} se observan
como se ubican unos componentes sobre otros. Los componentes se comunican con
aquellos adyacentes. El RTOS es un componente opcional y que funciona de forma
paralela a \extranjerismo{drivers} y \extranjerismo{middleware}.



\section{Familia de protocolos TCP/IP}{\label{sec:tcpip}}
Otra parte fundamental del proyecto consiste en dotar de conectividad al SE
usando TCP/IP. Para que dos dispositivos sean capaz de comunicarse necesitan
intercambiar datos de una forma conocida y consensuada entre ambos. Con el
objetivo de garantizar la interoperabilidad entre sistemas de diferentes modelos
o fabricantes surgen protocolos estandarizados que garantizan dicha
comunicación.

En \titulo{A dictionary of computing} \cite{butterfield2016} se define un
protocolo como:
\begin{quotation}``Un acuerdo que gobierna los procedimientos usados
  para intercambiar información entre entidades cooperantes. [...] La
  información es transferida a la ubicación remota usando el protocolo de más
  bajo nivel, después pasa de manera ascendente via interfaces hasta alcanzar el
  nivel correspondiente en el destino. En general, el protocolo regulará el
  formato de los mensajes, la generación de la información de comprobación, el
  control de flujo, además de las acciones a tomar ante el acontecimiento de
  errores. [...]''
\end{quotation}

Como se puede ver en la definición, los protocolos definen la manera en la que
se transmite información en un mismo nivel. Dependiendo del modelo usado existen
más o menos niveles intercambiando información. Por ejemplo, según el  modelo de
interconexión de sistemas abiertos \cite{itutx200}, también conocido como Modelo
OSI, existen 7 niveles o capas.


\subsection{Descripción}{\label{sec:tcpip-desc}}
En este proyecto se utiliza la familia de protocolos TCP/IP, la cual se compone 
de decenas de protocolos. Los protocolos se reparten en alguno de los siguientes
niveles: enlace, internet, transporte, aplicación.

En comparación con el modelo OSI, la primera capa (física) queda delegada al
\extranjerismo{hardware} y las tres últimas (sesión, presentación y aplicación),
se combinan en una sola.

Las capas se definieron en los \titulo{Request for Comments} (RFC)
\cite{rfc1122} y \cite{rfc1123}. Siguiendo un orden lógico de menor a mayor
nivel se describen de la siguiente manera \cite{kozierok2005}:
\begin{description}
  \item[Enlace] Esta capaz es la encargada de actuar como interfaz entre el
  resto de capas superiores y la red local subyacente. Algunas redes TCP/IP no 
  usan protocolos en este nivel. Por ejemplo, las redes Ethernet se encargan de
  manejar las capas uno y dos. En otras redes sin implementación de este nivel
  se usan protocolos como Serial Line Internet Protocol (SLIP) \cite{rfc1055} y
  Point-to-Point Protocol (PPP) \cite{rfc1661}.

  Además, en esta capa se especifican otros protocolos comunes como el Address
  Resolution Protocol (ARP) \cite{rfc826} encargado del descubrimiento de las
  direcciones de enlace, o Neighbor Discovery Protocol (NDP) \cite{rfc4861} que
  proporciona cierta funcionalidades más que ARP en redes que usan el
  Protocolo de Internet Version 6 (IPv6).

  \item[Internet] Nivel correspondiente a la capa de red del modelo OSI. Se
  encarga del direccionamiento lógico de los dispositivos, del empaquetado,
  manipulación y envío de los datos, y del enrutamiento.
  
  En esta capa de define el protocolo fundamental de todo el conjunto conocido
  como Internet Protocol (IP). Actualmente hay dos versiones en funcionamiento,
  la 4 y la 6. Para distinguir una de otra se usa la nomenclatura IPv4
  \cite{rfc791} y IPv6 \cite{rfc2460} respectivamente.

  Además, en esta capa existen otros protocolos de soporte. Internet Control
  Message Protocol para IPv4 (ICMP) \cite{rfc792} y IPv6 (ICMPv6) \cite{rfc4443}
  son usados para el control y la notificación del errores. Internet Group
  Management Protocol (IGMP) \cite{rfc4604} y Multicast Listener Discovery (MLD)
  \cite{rfc4604} se usan para establecer grupos de difusión múltiple.

  \item[Transporte] La principal labor realizada en esta capa es facilitar
  la comunicación lógica entre procesos de aplicación. La comunicación se puede
  orientar de dos maneras, la primera con protocolos orientados a conexión y la
  segunda con protocolos no orientados a conexión.

  Los protocolos fundamentales de esta capa son el Transmission Control Protocol
  (TCP) \cite{rfc793} y el User Datagram Protocol (UDP) \cite{rfc768}.
  TCP junto con IP son los protocolos más importantes de toda la familia como se
  puede deducir al ser usados en su nombre "TCP/IP".

  TCP es un protocolo orientado a la conexión, proporciona un conjunto completo
  de servicios para aquellas aplicaciones que lo necesiten y su uso se considera
  fiable. Con el uso de puertos consigue que varias aplicaciones puedan usar una
  misma dirección IP. El protocolo establece una conexión virtual entre dos
  dispositivos capaz de enviar información de manera bidireccional. Las
  transmisiones usan una ventana deslizante que permite detectar aquellas que no
  reconocidas para que sean retransmitidas.

  UDP es un protocolo no orientado a la conexión, más simple que TCP y sus
  transmisiones pueden resultar no fiables. Aunque también proporciona el uso
  de puertos como TCP, UDP no agrega nada más, no se establece una conexión este
  dispositivos, no se realizan retransmisiones, por lo tanto, determinadas
  transmisiones pueden llegar a perderse. Su uso se justifica en aquellos
  escenarios donde la velocidad de la transmisión prima por encima de todo
  aunque se incurra ocasionalmente en la perdida de información.

  \item[Aplicación] Esta es la cuarta y última capa del modelo TCP/IP, por ello 
  es la capa de mayor nivel. Los protocolos de esta capa tienen objetivos de
  índole más diversa que los protocolos que se encuentran en capas de menor
  nivel.

  Por ejemplo, con Dynamic Host Configuration Protocol (DHCP) \cite{rfc2131}
  un servidor DHCP puede asignar direcciones IP y otros parámetros de
  configuración a los dispositivos que lo soliciten. Domain Name System (DNS)
  \cite{rfc1034} \cite{rfc1035} resuelve o traduce entre direcciones IP y
  nombres inteligibles por los usuarios.

  Hay protocolos que sirven para la transferencia en la web como Hypertext
  Transfer Protocol (HTTP) \cite{rfc7230}, su segunda versión (HTTP/2)
  \cite{rfc7540}, o de forma segura Hypertext Transfer Protocol Secure (HTTPS) 
  \cite{rfc2818}. Protocolos para la recepción de correo electrónico como
  Internet Message Access Protocol (IMAP) \cite{rfc3501} o Post Office
  Protocol version 3 (POP3) \cite{rfc1939}, también para el envío con
  Simple Mail Transfer Protocol (SMTP) \cite{rfc5321}.

  Otros protocolos comunes son Network Time Protocol (NTP) \cite{rfc5905}
  para la sincronización de relojes en línea. Session Initiation Protocol
  (SIP) \cite{rfc3261} es usado para establecer sesiones en tiempo real típicas
  de aplicaciones de voz, vídeo o mensajería. Secure Shell (SSH) \cite{rfc4253}
  establece una conexión segura sobre redes inseguras, muy utilizado para
  iniciar y ejecutar sesiones remotas.

  Para terminar, dos protocolos usados habitualmente en sistemas empotrados o
  en dispositivos de Internet de las cosas (IoT) son Message Queuing Telemetry 
  Transport (MQTT) \cite{banks2015} y Constrained Application Protocol (CoAP)
  \cite{rfc7252}. Mientras que el primero se apoya en TCP y en un bróker para
  el intercambio de mensajes, el segundo emplea UDP y está caracterizado por su
  baja sobrecarga ideal para ser utilizado en dispositivos limitados.
\end{description}
